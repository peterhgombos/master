% !TEX encoding = UTF-8 Unicode
%!TEX root = thesis.tex
% !TEX spellcheck = en-US
%%=========================================
\chapter{Introduction}
An embedded system is a computer system which is embedded in the devices it controls, and is designed for a dedicated usage.
Some examples of systems dedicated to a specific purpose are mobile phones, sensors and ATMs.
By embedding a computer system, the hardware can be specifically tailored for the application of the system.
The application of the system also adds constraints, and the solutions may contradict other constraints
In a battery driven system, a really powerful battery that lasts a long time would be better, but the size of the battery might make it no longer suitable for using the system on the move.
Other constraints that are often imposed on the systems are memory size, storage space, maximum operating temperature and physical size.

Power constraints are one of the most prevalent in current embedded systems, as more and more devices are connected to the internet, and regular needs charging.
But some systems can’t be charged at regular intervals, for example sensor systems in areas that are hard to reach, on the bottom of the sea, embedded in concrete or someplace else; need to be able to function for a long time on a single battery.
As lower energy use also means lower heat dissipation, better power usage will lead to a cooler system, which is very important in handheld devices like mobile phones. 

In the early days of computing, the way the computers were programmed was with binary code, which is tedious to write, and requires the programmer to remember various arbitrary codes. 
Quite early, assembly languages were developed, which provides a one to one mapping of easier to remember function names to the machine code.
While far better for the programmer, assembly programming is still very tedious to write, and when creating a big complex language, the data models the assembler provides are too basic. 
Virtually all programs written today are written in a high-level programming language, which means they are not a direct mapping of machine functions, but code written in it need to be transformed into something the machine can run. 
This transformation can happen in two different ways, compilation or interpretation. 

A compiler is a program that takes source code in a given language as input, and outputs code in some other form. 
This can either be into a lower level language, like assembly, or into another programming language entirely.
This source-to-source compiling is also known as transcompiling, or simply transpiling.
The compiled program when run takes the input and produces an output. 
An interpreter, on the other hand, is a program running the code, and takes the input at the same time and produces the output. See figure.

While compilation and interpretation give a lot of advantages, they do add some overhead while running programs. 
A perfect compilation will translate the source code into an equivalent with no more instructions than needed, which is impossible except in trivial cases, as high-level languages adds many abstractions that are not easily translated to the basic data models of assembly languages. 
In the case of interpretation, the interpreter running on the computer by definition adds extra instructions. 
However, the computer can do things to the source code that are not natural for a programmer, like statistical analysis, which can lead to other optimizations like dead code elimination, loop unrolling etc.

JavaScript is a dynamic, weakly typed language, originally created for allowing interactivity on web pages.
Developed in 1995 by Netscape, it became the basis of the ECMAscript standard in 1997. (\cite{jshistory})
This standardization of JavaScript to work in every browser allowed it to become the most used programming language on the web.
Currently, almost 90\% of web pages uses JavaScript (\cite{jsclientstats}).
Newer developments in the JavaScript world is the node.js, later forked into io.js, a stand alone framework for running the V8 engine, allowing for the use of JavaScript on the server side.
In just three years, JavaScript is now used on 0.2\% of websites for server side code.(\cite{jsserverstats})

This dominance of JavaScript on the web has created to a large community around the language, with a lot of tools .
For some, there is a want to use one language to do everything, and so multiple projects for using JavaScript on embedded platforms have risen.
In addition to the benefits of a high-level language, there is the added bonus of not having to use time to learn a new programming language when developing for an embedded platform.

\section{What are some common ways of making a system more energy efficient?}
Look at Jahre's lectures from EEDS.

\section{Related work}
The first look at how you can optimize software to achieve better energy performance was done by \cite{tiwari94}.
Before their work, power measurement tools where only available at the circuit and logic level.
They then introduce a way of estimating the energy cost of the instructions that are available on the processor, by measuring the current into the microprocessor with an ammeter, and a model for estimating the power consumption of a program by looking at the instructions of the program.
With these data, they can optimize code running on the processor by using a strategy that minimizes the use of power hungry instructions.

In \cite{russell98}, finds that the model developed by \cite{tiwari94} is needlessly detailed, and that the power consumption of a program can be estimated within 8\% accuracy by using the average power consumption per cycle of the processor, multiplied with the execution time in cycles of the program.
Their conclusion then is that to optimizations that minimize execution time of a program, also reduces the energy consumption of the same program.

\cite{ortiz08} looked at some source code optimizations to find if they would lower the energy consumption, but found that the impact of the techniques varied on the platforms they tested against.
This device specific optimization was explored further by \cite{delima13}, where the authors try to find a set of compiler optimizations that gives the best result for the program.
They do this by reducing the number of possible sets of optimizations and test code compiled with the sets of optimizations, finding the set that provides the best performance gain or is most power efficient.
While their research focuses on a desktop computer, the same could be done for an embedded system.

Kavvadias

\section{Problem}
To create embedded systems, there’s often a need to use lower level languages, to achieve the desired power and speed requirements. It can be quite hard to program in these languages, with few abstractions and a need for the programmer to do a lot of dirty work. On the other hand, high-level languages offer many improvements for the programmer, as well as a familiar programming environment. Unfortunately, high-level language often mean a high level of energy use. To consolidate these directions, this thesis ask the question: “What are energy efficient ways of bringing high-level programming languages to embedded devices?”

This will be explored through comparing the energy use of some JavaScript engines, particularly with the Tessel 1, and a Raspberry Pi board running Espruino and io.js, while running benchmark code.

\section{Outline of this thesis}
In chapter \ref{ch:chapter2} the method of the experiment is explained, together with the rationale behind the experiment.
The results of the this experiment is presented in chapter \ref{ch:chapter3}, and these are discussed in chapter \ref{ch:chapter4}.
Concluding remarks are added in chapter \ref{ch:chapter5}.