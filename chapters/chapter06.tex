% !TEX encoding = UTF-8 Unicode
%!TEX root = thesis.tex
% !TEX spellcheck = en-US
%%=========================================
\chapter{Conclusion}
\label{chap:chapter6}

The tessel compiler can be optimized.

Code need to be tailored to Espruino

Focus on speed in implementations

Energy can be saved by improving the OS.

\section{Future work}
Run more tests with more instructions.

Run similar tests with more advanced programs.

Test tessel 2 when it comes out.

Test on Raspberry Pi 2

Run optimized benchmarks as laid out in \fref{sec:benchmarkdiscussion}

\subsection{Experiment automation on the Tessel}
Another way of automating the experiments on the Tessel than what was done, would be to connect it to a computer using another modified USB cable.
A USB cable carries it signal through four cable, two are for Vcc and ground, while the last two are for the data connection.
(There actually is a fifth cable in newer cables that can be used for control, but it is optional.)
There should be no problem in powering the device from an external power supply, as done in the experiment, and at the same time delivering data from a computer.

The benefits of this, is to get the same control over the programs run as through the shell of the Raspberry Pi.
This would remove the need for resetting the Tessel from the software itself, making it easier to see where the program runs in the sample data.

\todo{cite usb docs}However, there is no guarantee for this to work, as it is undocumented in the specification.
There should be no connection between the power and data lines when using a USB cable, but implementations might vary.
This needs to be tested to be done, and because of timing limitations in this project, the solution described in \fref{chap:chapter3} was decided to be good enough.

